% Compile with XeLaTeX, TeXLive 2013 or more recent
\documentclass{beamer}

% Base packages
\usepackage{fontspec}

\usepackage{xunicode}
\usepackage{xltxtra}

\usepackage{amsfonts}
\usepackage{amsmath}
\usepackage{longtable}
\usepackage{csquotes}
\usepackage{standalone}

\usepackage{graphicx}
\graphicspath{{./images/}}

\usepackage{tikz}
\usetikzlibrary{arrows,decorations.pathmorphing,backgrounds,positioning,fit,petri}

\usepackage{listings}
\lstset{language=C, basicstyle=\ttfamily, breaklines=true, keepspaces=true, keywordstyle=\color{blue}}

% Setup Russian hyphenation
\usepackage{polyglossia}
\setdefaultlanguage[spelling=modern]{russian} % for polyglossia
\setotherlanguage{english} % for polyglossia

% Setup fonts
\newfontfamily\russianfont{CMU Serif}
\setromanfont{CMU Serif}
\setsansfont{CMU Sans Serif}
\setmonofont{CMU Typewriter Text}

% Be able to insert hyperlinks
\usepackage{hyperref}
\hypersetup{colorlinks=true, linkcolor=black, filecolor=black, citecolor=black, urlcolor=blue , pdfauthor=Evgeny Yulyugin <yulyugin@gmail.com>, pdftitle=Параллельное программирование}
% \usepackage{url}

% Misc optional packages
\usepackage{underscore}
\usepackage{amsthm}

% A new command to mark not done places
\newcommand{\todo}[1][]{{\color{red}TODO\ #1}}

\newcommand{\abbr}{\textit{англ.}\ }

\subtitle{Курс «Параллельное программирование»}
\subject{Lecture}
\author[Евгений Юлюгин]{Евгений Юлюгин \\ \small{\href{mailto:yulyugin@gmail.com}{yulyugin@gmail.com}}}
\date{\today}
\pgfdeclareimage[height=0.5cm]{mipt-logo}{../common/mipt.png}
\logo{\pgfuseimage{mipt-logo}}

\typeout{Copyright 2014 Evgeny Yulyugin}

\usetheme{Berlin}
\setbeamertemplate{navigation symbols}{}%remove navigation symbols


\title{Общине вопросы. Состояние гонки. Примитивы синхронизации.}

\begin{document}

\begin{frame}
\titlepage
\end{frame}

\section*{Обзор}
\begin{frame}{На прошлой лекции}

\begin{itemize}
    \item Основы MPI.
\end{itemize}
\end{frame}

\begin{frame}{На этой лекции}
\tableofcontents
\end{frame}

\section{Определения}

\begin{frame}{Атомарные операции}
Атомарные операции (\abbr atomic operations) --- операции, выполняющиеся как
единое целое, либо не выполняющиеся вовсе.
\vspace*{0.5cm}

Доступ к разделяемым ресурсам должен быть атомарным.
\vspace*{0.5cm}

Не все инструкции процессора атомарны!
\end{frame}

\begin{frame}{Критическая секция}
Критическая секция (\abbr critical section) --- участок кода программы, в
котором производится доступ к общему ресурсу, которые не должен быть использован
более чем одним потоком исполнения.
\vfill\pause
Какие ресурсы могут быть критическими?
\pause
\begin{itemize}
    \item данные,
    \pause
    \item устройство.
\end{itemize}
\end{frame}

\section{Состояние гонки}

\begin{frame}{Состояние гонки}
Состояние гонки (\abbr Race condition) --- ошибка в многопоточной программе, при которой работа приложения зависит от того, в каком порядке выполняются части кода.
\vspace*{0.5cm}

Свое название получила от похожей ошибки проектирования электронных схем (Гонки сигналов).
\vspace*{0.5cm}

Состояние гонки --- ошибка проявляющаяся в случайный момент времени.
\end{frame}

\begin{frame}[fragile]{Пример}
\begin{lstlisting}[language=C++,basicstyle=\ttfamily,keywordstyle=\color{blue},basicstyle=\scriptsize]
int N = 1000;
int x = 0;
\end{lstlisting}

\begin{columns}[t]
    \begin{column}[T]{0.45\textwidth}
    \begin{lstlisting}[language=C++,basicstyle=\ttfamily,keywordstyle=\color{blue},basicstyle=\scriptsize]
// thread 0
for (i = 0; i < N; ++i) {
    ++x;
}
    \end{lstlisting}
    \end{column}
    \begin{column}[T]{0.45\textwidth}
    \begin{lstlisting}[language=C++,basicstyle=\ttfamily,keywordstyle=\color{blue},basicstyle=\scriptsize]
// thread 1
for (i = 0; i < N; ++i) {
    if (x%2 == 0)
        printf("%d\n", x%2);
}
    \end{lstlisting}
    \end{column}
\end{columns}
\end{frame}

\begin{frame}{Критерии избежания состояния гонки}
\begin{itemize}
    \item Два процесса не должны одновременно находиться в критических областях,
    \item В программе не должно быть предположения о скорости и количестве
    процессоров,
    \item Процесс, находящийся вне критической области, не может блокировать
    другие процессы,
    \item Невозможно ситуация, когда процесс вечно ждет попадания в критическую
    секцию.
\end{itemize}
\end{frame}

\section{Синхронизация}

\begin{frame}{Семафор}
Семафор --- объект, ограничивающий количество потоков, которые могут войти в заданный участок кода.
\vfill
Интерфейс семафора:
\begin{itemize}
    \item init(n) --- установить счетчик в n,
    \item enter() --- подождать пока счетчик не станет больше нуля, затем уменьшить его на единицу,
    \item leave() --- увеличить счетчик на еденицу.
\end{itemize}
\end{frame}

\begin{frame}{Мьютекс}
Мьютекс (\abbr mutex, mutual exclusion --- <<взаимное исключение>>) --- <<одноместный>> семафор, служащий для синхранизации одновременно выполняющихся потоков.
\vfill
\begin{itemize}
    \item plain --- нет контроля повторного захвата тем же потоком;
    \item recursive --- повторные захваты тем же модулем допустимы, ведется
    счетчик захватов;
    \item timed --- нет контроля захвата тем же потоком, поддерживается захват
    мьютекса с тайм-аутом;
    \item recursive\_timed --- повторные захваты тем же модулем допустимы,
    поддерживается захват с тайм-аутом.
\end{itemize}
\end{frame}

\begin{frame}[fragile]{Пример}
\begin{lstlisting}[language=C++,basicstyle=\ttfamily,keywordstyle=\color{blue},basicstyle=\scriptsize]
int N = 1000;
int x = 0;
pthread_mutex_t mutex = PTHREAD_MUTEX_INITIALIZER;
\end{lstlisting}

\begin{columns}[t]
    \begin{column}[T]{0.5\textwidth}
    \begin{lstlisting}[language=C++,basicstyle=\ttfamily,keywordstyle=\color{blue},basicstyle=\scriptsize]
// thread 0
for (i = 0; i < N; ++i) {
    pthread_mutex_lock(&mutex);
    ++x;
    pthread_mutex_unlock(&mutex);
}
    \end{lstlisting}
    \end{column}
    \begin{column}[T]{0.5\textwidth}
    \begin{lstlisting}[language=C++,basicstyle=\ttfamily,keywordstyle=\color{blue},basicstyle=\scriptsize]
// thread 1
for (i = 0; i < N; ++i) {
    pthread_mutex_lock(&mutex);
    if (x%2 == 0)
        printf("%d\n", x%2);
    pthread_mutex_unlock(&mutex);
}
    \end{lstlisting}
    \end{column}
\end{columns}
\end{frame}

\section{Программная реализация}

\begin{frame}[fragile]{Алгоритм Деккера}
\begin{lstlisting}[language=C++,basicstyle=\ttfamily,keywordstyle=\color{blue},basicstyle=\scriptsize]
bool flag[2] = {false, false};
bool turn = 0;
\end{lstlisting}

\begin{columns}[t]
    \begin{column}[T]{0.45\textwidth}
        \begin{lstlisting}[language=C++,basicstyle=\ttfamily,keywordstyle=\color{blue},basicstyle=\scriptsize]
// thread 0
flag[0] = true;
while (flag[1]) {
    if (turn) {
        flag[0] = false;
        while (turn);
        flag[0] = true;
    }
}

// critical section
//...
turn = true;
flag[0] = false;
// end of critical section
// ...
        \end{lstlisting}
    \end{column}
    \begin{column}[T]{0.45\textwidth}
        \begin{lstlisting}[language=C++,basicstyle=\ttfamily,keywordstyle=\color{blue},basicstyle=\scriptsize]
// thread 1
flag[1] = true;
while (flag[0]) {
    if (!turn) {
        flag[1] = false;
        while (!turn);
        flag[1] = true;
    }
}

// critical section
//...
turn = false;
flag[1] = false;
// end of critical section
// ...
        \end{lstlisting}
    \end{column}
\end{columns}
\end{frame}

\begin{frame}{Алгоритм Деккера}
Особенности:
\begin{itemize}
    \item Не требует поддержки атомарных операций чтения/записи $\Rightarrow$
    переносимость;
    \item Применим только для случая с двумя процессами;
    \item Использование ждущего цикла (\abbr busy loop);
    \item Не работает на суперскалярных процессорах.
\end{itemize}
\end{frame}

\begin{frame}[fragile]{Алгоритм Петерсона}
\begin{lstlisting}[basicstyle=\small]
bool interested[2];
int turn;

void enter_region(int thread_id) {
    int other = 1 - thread_id;
    interested[thread_it] = true;
    turn = other;

    while(turn == other && interested[other]);
}

void leave_region(int thread_id) {
    interested[thread_id] = false;
}
\end{lstlisting}
\end{frame}

\begin{frame}[fragile]{Spinlock}
Spinlock --- низкоуровневый примитив синхронизации.
\vfill
Реализация:
\begin{lstlisting}[language={[x86masm]Assembler}, basicstyle=\scriptsize]
mov eax, spinlock_address
mov ebx, SPINLOCK_BUSY

wait_cycle:
lock xchg [eax], ebx
cmp ebx, SPINLOCK_FREE
jmp waait_cycle

; critical section

mov eax, spinlock_address
mov ebx, SPINLOCK_FREE
lock xchg [eax], ebx
\end{lstlisting}
\end{frame}

\section{Ошибки}

\begin{frame}[fragile]{Взаимная блокировка}
Взаимная блокировка (\abbr deadlock) --- ситуация в многозадачной среде, при которой несколько процессов находятся в состоянии бесконечного ожидания ресурсов, занятых самими этими процессами.
\vfill
\begin{lstlisting}[language=C++,basicstyle=\ttfamily,keywordstyle=\color{blue},basicstyle=\footnotesize]
pthread_mutex_t A, B;
pthread_mutex_init(&A, NULL);
pthread_mutex_init(&B, NULL);
\end{lstlisting}

\begin{columns}[t]
    \begin{column}[T]{0.45\textwidth}
        \begin{lstlisting}[language=C++,basicstyle=\ttfamily,keywordstyle=\color{blue},basicstyle=\scriptsize]
// thread 0
pthread_mutex_lock(&A);
// ...
pthread_mutex_lock(&B);
// ...
        \end{lstlisting}
    \end{column}
    \begin{column}[T]{0.45\textwidth}
        \begin{lstlisting}[language=C++,basicstyle=\ttfamily,keywordstyle=\color{blue},basicstyle=\scriptsize]
// thread 1
pthread_mutex_lock(&B);
// ...
pthread_mutex_lock(&A);
// ...
        \end{lstlisting}
    \end{column}
\end{columns}
\end{frame}

\begin{frame}{Livelock}
Ситуация, когда в отличие от обычной блокировки процессы не зависают, а занимаются бесполезной работой.
\vspace*{0.5cm}

Состояние системы постоянно меняется, но при этом она <<зациклилась>> и не производит полезной работы.
\end{frame}

\section*{Конец}

\begin{frame}[allowframebreaks]{Рекомендуемая литература}
\begin{thebibliography}{99}
    \bibitem{} \textit{Dijkstra~E.W.}
    \href{http://www.cs.utexas.edu/users/EWD/transcriptions/EWD01xx/EWD123.html}{Cooperating
    Sequential Processes}. --- 1965.
    \bibitem{} \textit{Tanenbaum~A.S.} Modern operating systems. --- 3rd~ed.
\end{thebibliography}
\end{frame}

\begin{frame}{Задания}

Вычислить $\sum \limits_{n=0}^{N} \frac{1}{n!}$.

N --- аргумент командной строки.

Построить графики времени работы в зависимости от количества процессов и ускорения в зависимости от количества процессов.

\end{frame}

\begin{frame}{На следующей лекции}

\begin{itemize}
    \item Классификация Флинна
    \item Память в параллельных вычислительных системах:
    \begin{itemize}
        \item Симметричная мультипроцессорность,
        \item Массово-параллельная архитектура,
        \item Архитектура с неравномерной памятью.
    \end{itemize}
\end{itemize}

\end{frame}

\begin{frame}

{\huge{Спасибо за внимание!}\par}

\vfill

\tiny{\textit{Замечание}: все торговые марки и логотипы, использованные в данном материале, являются собственностью их владельцев. Представленная здесь точка зрения отражает личное мнение автора, не выступающего от лица какой-либо организации.}

\end{frame}

\end{document}
