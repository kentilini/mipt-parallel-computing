% Compile with XeLaTeX, TeXLive 2013 or more recent
\documentclass{beamer}

% Base packages
\usepackage{fontspec}

\usepackage{xunicode}
\usepackage{xltxtra}

\usepackage{amsfonts}
\usepackage{amsmath}
\usepackage{longtable}
\usepackage{csquotes}
\usepackage{standalone}

\usepackage{graphicx}
\graphicspath{{./images/}}

\usepackage{tikz}
\usetikzlibrary{arrows,decorations.pathmorphing,backgrounds,positioning,fit,petri}

\usepackage{listings}
\lstset{language=C, basicstyle=\ttfamily, breaklines=true, keepspaces=true, keywordstyle=\color{blue}}

% Setup Russian hyphenation
\usepackage{polyglossia}
\setdefaultlanguage[spelling=modern]{russian} % for polyglossia
\setotherlanguage{english} % for polyglossia

% Setup fonts
\newfontfamily\russianfont{CMU Serif}
\setromanfont{CMU Serif}
\setsansfont{CMU Sans Serif}
\setmonofont{CMU Typewriter Text}

% Be able to insert hyperlinks
\usepackage{hyperref}
\hypersetup{colorlinks=true, linkcolor=black, filecolor=black, citecolor=black, urlcolor=blue , pdfauthor=Evgeny Yulyugin <yulyugin@gmail.com>, pdftitle=Параллельное программирование}
% \usepackage{url}

% Misc optional packages
\usepackage{underscore}
\usepackage{amsthm}

% A new command to mark not done places
\newcommand{\todo}[1][]{{\color{red}TODO\ #1}}

\newcommand{\abbr}{\textit{англ.}\ }

\subtitle{Курс «Параллельное программирование»}
\subject{Lecture}
\author[Евгений Юлюгин]{Евгений Юлюгин \\ \small{\href{mailto:yulyugin@gmail.com}{yulyugin@gmail.com}}}
\date{\today}
\pgfdeclareimage[height=0.5cm]{mipt-logo}{../common/mipt.png}
\logo{\pgfuseimage{mipt-logo}}

\typeout{Copyright 2014 Evgeny Yulyugin}

\usetheme{Berlin}
\setbeamertemplate{navigation symbols}{}%remove navigation symbols


\title{Анализ зависимостей}

\begin{document}

\begin{frame}
\titlepage
\end{frame}

\section*{Обзор}

\begin{frame}{На прошлой лекции}
\end{frame}

\begin{frame}{На этой лекции}
\tableofcontents
\end{frame} 

\section{Виды зависимостей}

\begin{frame}{Виды зависимостей}
\begin{enumerate}
    \item Зависимости по данным,
    \pause
    \item Зависимости по управлению,
    \pause
    \item Зависимости по ресурсам.
\end{enumerate}
\end{frame}

\section{Зависимости по данным}

\begin{frame}{Зависимости по данным}
Виды зависимостей по данным:

\begin{itemize}
    \item Зависимость по выходным данным,
    \item Зависимость по входным данным,
    \item Антизависимость,
    \item Истинная зависимость.
\end{itemize}
\end{frame}

\begin{frame}{Зависимости по данным}
\begin{itemize}
  \item WAW --- Зависимость по выходным данным,
  \item RAR --- Ложная зависимость,
  \item WAR --- Антизависимость,
  \item RAW --- Истинная зависимость.
\end{itemize}
\end{frame}

\begin{frame}[fragile]{Зависимость по выходным данным}
Output dependence.

\pause\bigskip

\begin{lstlisting}
x = 2 * y + 5;
x = 3 + k;
\end{lstlisting}

Мешает ли распараллеливанию?

\pause\bigskip

\begin{lstlisting}
x1 = 2 * y + 5;
x2 = 3 + k;
\end{lstlisting}
\end{frame}

\begin{frame}[fragile]{Зависимость по входным данным}
Input dependence.

\pause\bigskip

\begin{lstlisting}
y = x + 4;
z = x + 5;
\end{lstlisting}
\end{frame}

\begin{frame}[fragile]{Антизависимость}
Antidependence.

\pause\bigskip

\begin{lstlisting}
x = 2 * y + 1
y = z + 2
\end{lstlisting}

Мешает ли распараллеливанию?

\pause

\begin{lstlisting}
y1 = y
x = 2 * y1 + 1
y = z + 2
\end{lstlisting}
\end{frame}

\begin{frame}[fragile]{Истинная зависимость}
Flow (True) dependence.

\pause\bigskip

\begin{lstlisting}
x = 2 + z
y = 4 + x
\end{lstlisting}

\pause\bigskip

Мешает ли распараллеливанию?

\pause\bigskip

Да
\end{frame}

\begin{frame}[fragile]

\begin{lstlisting}
for (i = 0; i < M; ++i) {
  a[i] = f(a[i])
}
\end{lstlisting}

можно распараллелить на $M$ потоков.

\end{frame}

\begin{frame}[fragile]

\begin{lstlisting}
for (i = 0; i < M - 1; ++i) {
  a[i] = f(a[i + 1]);
}
\end{lstlisting}

антизависимость $=>$ можно распараллелить на $M$.

\end{frame}

\begin{frame}[fragile]

\begin{lstlisting}
for (i = 1; i < M; ++i) {
  a[i] = f(a[i-1]);
}
\end{lstlisting}

нельзя распараллелить

\end{frame}

\begin{frame}[fragile]

\begin{lstlisting}
for (i = 0; i < M; ++i) {
  a[i] = f(a[i - 2]);
}
\end{lstlisting}

можно распараллелить на два потока

\end{frame}

\begin{frame}[fragile]

\begin{lstlisting}
do {
  A[f(k)] = ...;
  ... = A[g(l)];
}
\end{lstlisting}

Если $f(k) == g(l)$, то распараллелить нельзя, так как производится чтение и запись одного и того же элемента.

\end{frame}

\begin{frame}[fragile]

Для того, чтоб узнать, можно ли распараллелить цикл необходимо вычислить

$D = K - L$

\begin{lstlisting}
for (i = 0; i < M; ++i) {
  a[i] = f(a[i+1]);
}
\end{lstlisting}

Разворачиваем цикл

\begin{itemize}
  \item \texttt{a[1] = f(a[2])}
  \item \texttt{a[2] = f(a[3])}
\end{itemize}

D = -1 $=>$ в цикле есть антизависимость и от нее можно избавиться копирование данных

\end{frame}

\section*{Конец}
% The final "thank you" frame 

\begin{frame}{Задания}
\end{frame}

\begin{frame}{На следующей лекции}
\end{frame}

\begin{frame}

{\huge{Спасибо за внимание!}\par}

\vfill

\tiny{\textit{Замечание}: все торговые марки и логотипы, использованные в данном материале, являются собственностью их владельцев. Представленная здесь точка зрения отражает личное мнение автора, не выступающего от лица какой-либо организации.}

\end{frame}

\end{document}
