% Compile with XeLaTeX, TeXLive 2013 or more recent
\documentclass{beamer}

% Base packages
\usepackage{fontspec}

\usepackage{xunicode}
\usepackage{xltxtra}

\usepackage{amsfonts}
\usepackage{amsmath}
\usepackage{longtable}
\usepackage{csquotes}
\usepackage{standalone}

\usepackage{graphicx}
\graphicspath{{./images/}}

\usepackage{tikz}
\usetikzlibrary{arrows,decorations.pathmorphing,backgrounds,positioning,fit,petri}

\usepackage{listings}
\lstset{language=C, basicstyle=\ttfamily, breaklines=true, keepspaces=true, keywordstyle=\color{blue}}

% Setup Russian hyphenation
\usepackage{polyglossia}
\setdefaultlanguage[spelling=modern]{russian} % for polyglossia
\setotherlanguage{english} % for polyglossia

% Setup fonts
\newfontfamily\russianfont{CMU Serif}
\setromanfont{CMU Serif}
\setsansfont{CMU Sans Serif}
\setmonofont{CMU Typewriter Text}

% Be able to insert hyperlinks
\usepackage{hyperref}
\hypersetup{colorlinks=true, linkcolor=black, filecolor=black, citecolor=black, urlcolor=blue , pdfauthor=Evgeny Yulyugin <yulyugin@gmail.com>, pdftitle=Параллельное программирование}
% \usepackage{url}

% Misc optional packages
\usepackage{underscore}
\usepackage{amsthm}

% A new command to mark not done places
\newcommand{\todo}[1][]{{\color{red}TODO\ #1}}

\newcommand{\abbr}{\textit{англ.}\ }

\subtitle{Курс «Параллельное программирование»}
\subject{Lecture}
\author[Евгений Юлюгин]{Евгений Юлюгин \\ \small{\href{mailto:yulyugin@gmail.com}{yulyugin@gmail.com}}}
\date{\today}
\pgfdeclareimage[height=0.5cm]{mipt-logo}{../common/mipt.png}
\logo{\pgfuseimage{mipt-logo}}

\typeout{Copyright 2014 Evgeny Yulyugin}

\usetheme{Berlin}
\setbeamertemplate{navigation symbols}{}%remove navigation symbols


\title{Разработка многопоточных приложений на Java.}

\begin{document}

\begin{frame}
\titlepage
\end{frame}

\section*{Обзор}

\begin{frame}{На прошлой лекции}
\begin{itemize}
\ifsbertech
    \item Атомарные операции,
    \item Состояние гонки,
    \item Примитивы синхронизации,
    \item Программная реализация,
    \item Популярные ошибки.
\fi
\end{itemize}
\end{frame}

\begin{frame}{На этой лекции}
\tableofcontents
\end{frame}

\section{Атомарные типы}

\begin{frame}{Атомарные типы}
Пакет java.util.concurrent.atomic
\vfill
\begin{itemize}
    \item AtomicBoolean,
    \item AtomicInteger,
    \item AtomicLong,
    \item AtomicReference<Type>.
\end{itemize}
\vfill
Операции:
\begin{itemize}
    \item Type get();
    \item void set(Type newValue);
    \item boolean compareAndSet(Type expect, Type update);
\end{itemize}
\end{frame}

\begin{frame}[fragile]{Атомарные операции}
Метод compareAndSet позволяет реализовывать другие операции.
\vfill
Пример из java.util.concurrent.atomic.AtomicInteger
\begin{lstlisting}
public final int getAndDecrement() {
        for (;;) {
            int current = get();
            int next = current - 1;
            if (compareAndSet(current, next))
                return current;
        }
    }
\end{lstlisting}
\end{frame}

\section{Примитивы синхронизации}

\section{Коллекции}

\section*{Конец}

\begin{frame}[allowframebreaks]{Рекомендуемая литература}
\begin{thebibliography}{99}
    \bibitem{} Обзор java.util.concurrent.
    \url{http://habrahabr.ru/company/luxoft/blog/157273}.
    \bibitem{} The Java Tutorials.
    \href{http://docs.oracle.com/javase/tutorial/essential/concurrency/index.html}{Lesson: Concurrency.}
    \bibitem{} \href{http://docs.oracle.com/javase/7/docs/api/java/util/concurrent/package-summary.html}{Package java.util.concurrent}.
\end{thebibliography}
\end{frame}

\begin{frame}{Задания}
\end{frame}

\begin{frame}{На следующей лекции}
\begin{itemize}
    \item Классификации параллельных вычислительных систем:
    \begin{itemize}
        \item Классификация Флинна,
        \item Классификация Хокни,
    \end{itemize}
    \item Суперскалярная архитектура,
    \item VLIW арфитектура.
\end{itemize}
\end{frame}

\begin{frame}

{\huge{Спасибо за внимание!}\par}

\vfill

\tiny{\textit{Замечание}: все торговые марки и логотипы, использованные в данном материале, являются собственностью их владельцев. Представленная здесь точка зрения отражает личное мнение автора, не выступающего от лица какой-либо организации.}

\end{frame}

\end{document}
