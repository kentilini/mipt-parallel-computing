% Compile with XeLaTeX, TeXLive 2013 or more recent
\documentclass{beamer}

% Base packages
\usepackage{fontspec}

\usepackage{xunicode}
\usepackage{xltxtra}

\usepackage{amsfonts}
\usepackage{amsmath}
\usepackage{longtable}
\usepackage{csquotes}
\usepackage{standalone}

\usepackage{graphicx}
\graphicspath{{./images/}}

\usepackage{tikz}
\usetikzlibrary{arrows,decorations.pathmorphing,backgrounds,positioning,fit,petri}

\usepackage{listings}
\lstset{language=C, basicstyle=\ttfamily, breaklines=true, keepspaces=true, keywordstyle=\color{blue}}

% Setup Russian hyphenation
\usepackage{polyglossia}
\setdefaultlanguage[spelling=modern]{russian} % for polyglossia
\setotherlanguage{english} % for polyglossia

% Setup fonts
\newfontfamily\russianfont{CMU Serif}
\setromanfont{CMU Serif}
\setsansfont{CMU Sans Serif}
\setmonofont{CMU Typewriter Text}

% Be able to insert hyperlinks
\usepackage{hyperref}
\hypersetup{colorlinks=true, linkcolor=black, filecolor=black, citecolor=black, urlcolor=blue , pdfauthor=Evgeny Yulyugin <yulyugin@gmail.com>, pdftitle=Параллельное программирование}
% \usepackage{url}

% Misc optional packages
\usepackage{underscore}
\usepackage{amsthm}

% A new command to mark not done places
\newcommand{\todo}[1][]{{\color{red}TODO\ #1}}

\newcommand{\abbr}{\textit{англ.}\ }

\subtitle{Курс «Параллельное программирование»}
\subject{Lecture}
\author[Евгений Юлюгин]{Евгений Юлюгин \\ \small{\href{mailto:yulyugin@gmail.com}{yulyugin@gmail.com}}}
\date{\today}
\pgfdeclareimage[height=0.5cm]{mipt-logo}{../common/mipt.png}
\logo{\pgfuseimage{mipt-logo}}

\typeout{Copyright 2014 Evgeny Yulyugin}

\usetheme{Berlin}
\setbeamertemplate{navigation symbols}{}%remove navigation symbols


\title{Основы MPI}

\begin{document}

\begin{frame}
\titlepage
\end{frame}

\section*{Обзор}

\begin{frame}{На этой лекции}
\tableofcontents
\end{frame}

\section{История MPI}

\begin{frame}{Интерфейс передачи сообщений}
Интерфейс передачи сообщений (\abbr Massage Passing Interface, \texttt{MPI}) --- программный интерфейс, позволяющий обмениваться сообщениями между процессами выполняющими одну задачу.
\vfill
\begin{itemize}
    \item Разработан Уильямом Гроуппом, Эвином Ласком и другими;
    \item Первая версия разрабатывалась в 1993-1994 гг;
    \item \texttt{MPI} 1.1 опубликован в 12 июня 1995 года. Поддерживается
    большинством современных реализаций \texttt{MPI}. Первая раелизация
    появилась в 2002 году;
    \item Текущая версия --- \texttt{MPI} 3.0. 2012~г.;
    \item Существуют реализации для языков Fortran, Java, C и C++.
\end{itemize}
\end{frame}

\begin{frame}{Цель MPI}
Основная цель:
\begin{itemize}
    \item Переносимость кода,
    \item Эффективная реализация,
    \item Поддержка архитектур с неоднородной памятью.
\end{itemize}
\end{frame}

\section{Базовые функции MPI}

\begin{frame}[fragile]{Инициализация и завершение процессов}
Определены в заголовочном файле \texttt{mpi.h}
\vfill
\begin{lstlisting}
int MPI_Init(int *pargc, char ***pargv);
int MPI_Finalize(void);
\end{lstlisting}
\end{frame}

\begin{frame}{Коммуникатор MPI}
\begin{itemize}
    \item Группа процессов, которые могут взаимодействовать друг с другом,
    \item Все обращения к MPI функциям содерат коммуникатор, как параметр,
    \vfill
    \item Виды коммуникаторов:
    \begin{itemize}
        \item \texttt{MPI_COMM_WORLD} --- Создается при вызове \texttt{MPI_Init}.
        Содержит все процессы программы.
        \item \texttt{MPI_COMM_SELF} --- Содержит только сам процесс,
        \item Пользовательские коммуникаторы.
    \end{itemize}
\end{itemize}
\end{frame}

\begin{frame}[fragile]{Self-identification}

\begin{lstlisting}
MPI_Comm_size(MPI_Comm comm, int *size);
\end{lstlisting}

Returns the size of the group associated with communicator.

\vfill

\begin{lstlisting}
MPI_Comm_rank(MPI_Comm comm, int *rank);
\end{lstlisting}

Determines the rank of the calling process in the communicator.

\end{frame}

\begin{frame}[fragile]{Send/Receive Overview}

\begin{lstlisting}
int MPI_Send(void *buf, int count, MPI_Datatype datatype, int dest, int tag, MPI_Comm comm);
int MPI_Recv(void *buf, int count, MPI_Datatype datatype, int source, int tag, MPI_Comm comm, MPI_Status status);
\end{lstlisting}

\end{frame}

\begin{frame}{MPI_Datatype}

\begin{itemize}
    \item \texttt{MPI_CHAR},
    \item \texttt{MPI_INT},
    \item \texttt{MPI_UNSIGNED},
    \item \texttt{MPI_DOUBLE},
    \item etc.,
    \item Custom data types.
\end{itemize}

\end{frame}

\begin{frame}{MPI_Status \& tag}

\texttt{MPI_Status}:

\begin{itemize}
    \item Содержит дополнительную информацию о полученном сообщении.
    \item \texttt{MPI_STATUS_IGNORE} --- специальное значение, которое уменьшает количество используемой памяти в том случае, если вы не собираетесь рассматривать это поле.
\end{itemize}

\vfill

tag:

\begin{itemize}
    \item Число для идентификации сообщения,
    \item \texttt{MPI_ANY_TAG}.
\end{itemize}

\end{frame}

\begin{frame}{Блокирующие и не блокирующие операции}

\texttt{MPI_Send/Recv} --- блокирующиеся функции.

\begin{figure}
\centering
\begin{tikzpicture}
    \fill[red] (0,0) circle (0.15\textwidth);
    \fill[white] (-0.1\textwidth,-0.03\textwidth) rectangle (0.1\textwidth,0.03\textwidth);
\end{tikzpicture}
\end{figure}

\texttt{MPI_Isend/Irecv} --- не блокирующиеся варианты \texttt{MPI_Send/Recv}.

\end{frame}

\begin{frame}[fragile]{Вычисление времени}

\begin{lstlisting}
double MPI_Wtime()
\end{lstlisting}

Возвращает время в секундах, прошедшее с  некоторого момента в прошлом (точки отсчета). Гарантируется, что эта точка отсчета не будет изменена в течение жизни процесса.

\begin{lstlisting}
double MPI_Wtick()
\end{lstlisting}

Возвращает разрешение таймера (минимальное значение кванта времени).

\begin{lstlisting}
int MPI_Barrier(MPI_Comm comm)
\end{lstlisting}

Блокирует работу вызвавшего ее процесса до тех пор, пока все другие процессы группы также не вызовут эту функцию.

\end{frame}

\section{Компиляция и запуск}

\begin{frame}[fragile]{Компиляция и запуск}

Для компиляции используется \texttt{mpicc} вместо \texttt{gcc}.

\vfill

Запуск производится командой вида

\begin{lstlisting}
> mpirun –np <n_processes> <programm> <args>
\end{lstlisting}

\end{frame}

\section*{Конец}

\begin{frame}[allowframebreaks]{Рекомендуемая литература}
\begin{thebibliography}{99}
    \bibitem{} MPI Documents. \url{http://www.mpi-forum.org/docs/docs.html}
    \bibitem{} MPI standart. \url{http://www.mcs.anl.gov/research/projects/mpi}
    \bibitem{} \textit{Оленев~н.н.} Основы параллельного программирования в
    системе MPI. --- 2005 --- 81~с.
\end{thebibliography}
\end{frame}

\begin{frame}{Задания}

\begin{itemize}
    \item MPI "Hello, World!",
    \item MPI ring,
    \item MPI ping-pong.
\end{itemize}

\end{frame}

\begin{frame}{На следующей лекции}

\begin{itemize}
    \item Общая теория параллельного программирования:
    \begin{itemize}
        \item Состояние гонки,
        \item Примитивы синхронизации,
        \item Эффективнось работы параллельного алгоритма,
        \item Закон Амдала.
    \end{itemize}
\end{itemize}

\end{frame}

\begin{frame}

{\huge{Спасибо за внимание!}\par}

\vfill

\tiny{\textit{Замечание}: все торговые марки и логотипы, использованные в данном материале, являются собственностью их владельцев. Представленная здесь точка зрения отражает личное мнение автора, не выступающего от лица какой-либо организации.}

\end{frame}

\end{document}
