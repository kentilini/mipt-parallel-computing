% Compile with XeLaTeX, TeXLive 2013 or more recent
\documentclass{beamer}

% Base packages
\usepackage{fontspec}

\usepackage{xunicode}
\usepackage{xltxtra}

\usepackage{amsfonts}
\usepackage{amsmath}
\usepackage{longtable}
\usepackage{csquotes}
\usepackage{standalone}

\usepackage{graphicx}
\graphicspath{{./images/}}

\usepackage{tikz}
\usetikzlibrary{arrows,decorations.pathmorphing,backgrounds,positioning,fit,petri}

\usepackage{listings}
\lstset{language=C, basicstyle=\ttfamily, breaklines=true, keepspaces=true, keywordstyle=\color{blue}}

% Setup Russian hyphenation
\usepackage{polyglossia}
\setdefaultlanguage[spelling=modern]{russian} % for polyglossia
\setotherlanguage{english} % for polyglossia

% Setup fonts
\newfontfamily\russianfont{CMU Serif}
\setromanfont{CMU Serif}
\setsansfont{CMU Sans Serif}
\setmonofont{CMU Typewriter Text}

% Be able to insert hyperlinks
\usepackage{hyperref}
\hypersetup{colorlinks=true, linkcolor=black, filecolor=black, citecolor=black, urlcolor=blue , pdfauthor=Evgeny Yulyugin <yulyugin@gmail.com>, pdftitle=Параллельное программирование}
% \usepackage{url}

% Misc optional packages
\usepackage{underscore}
\usepackage{amsthm}

% A new command to mark not done places
\newcommand{\todo}[1][]{{\color{red}TODO\ #1}}

\newcommand{\abbr}{\textit{англ.}\ }

\subtitle{Курс «Параллельное программирование»}
\subject{Lecture}
\author[Евгений Юлюгин]{Евгений Юлюгин \\ \small{\href{mailto:yulyugin@gmail.com}{yulyugin@gmail.com}}}
\date{\today}
\pgfdeclareimage[height=0.5cm]{mipt-logo}{../common/mipt.png}
\logo{\pgfuseimage{mipt-logo}}

\typeout{Copyright 2014 Evgeny Yulyugin}

\usetheme{Berlin}
\setbeamertemplate{navigation symbols}{}%remove navigation symbols


\title{Коллективные операции в MPI}

\begin{document}

\begin{frame}
\titlepage
\end{frame}

\section*{Обзор}

\begin{frame}{На прошлой лекции}
\end{frame}

\begin{frame}{На этой лекции}
\tableofcontents
\end{frame} 

% Use [fragile] option to insert listings
% \begin{frame}[fragile]

\section{Широковещательная рассылка}

\begin{frame}[fragile]{MPI_Bcast}

\begin{lstlisting}
int MPI_Bcast(void *buffer, int count, MPI_Datatype datatype, int root, MPI_Comm comm);
\end{lstlisting}

\end{frame}

\begin{frame}{MPI_Bcast}

\begin{figure}
    \centering
    \begin{tikzpicture}
        \node[] (root) {Процесс root};
        \node[below = 2.5cm of root] (all) {Все процессы};

        \node[draw, fill=orange, rectangle, right= 3.5cm of root, minimum width= 2cm, minimum height= 0.5cm] (memroot) {};
        \draw[->] ([yshift= 0.3cm] memroot.north west) -- (memroot.north west);
        \node[above left= 0cm and -0.1cm of memroot.north] {\scriptsize{\texttt{buffer}}};
        \draw[<->] ([yshift= -0.2cm] memroot.south west) -- ([yshift= -0.2cm] memroot.south east);
        \node[fill=white, below= 0cm of memroot.south] {\scriptsize{\texttt{count}}};

        \node[draw, fill=orange, rectangle, right= 1cm of all, minimum width= 2cm, minimum height= 0.5cm] (proc0) {};
        \draw[->] ([yshift= 0.3cm] proc0.north west) -- (proc0.north west);
        \node[above left= 0cm and -0.1cm of proc0.north] {\scriptsize{\texttt{buffer}}};
        \draw[<->] ([yshift= -0.2cm] proc0.south west) -- ([yshift= -0.2cm] proc0.south east);
        \node[fill=white, below= 0cm of proc0.south] {\scriptsize{\texttt{count}}};
        \node[below= 0.3cm of proc0.south] {\scriptsize{процесс 0}};
        \node[right= 0cm of proc0, minimum width= 0cm, minimum height= 0.5cm, align=center] (etc0) {...};

        \node[draw, fill=orange, rectangle, right= 0cm of etc0, minimum width= 2cm, minimum height= 0.5cm] (proc1) {};
        \draw[->] ([yshift= 0.3cm] proc1.north west) -- (proc1.north west);
        \node[above left= 0cm and -0.1cm of proc1.north] {\scriptsize{\texttt{buffer}}};
        \draw[<->] ([yshift= -0.2cm] proc1.south west) -- ([yshift= -0.2cm] proc1.south east);
        \node[fill=white, below= 0cm of proc1.south] {\scriptsize{\texttt{count}}};
        \node[below= 0.3cm of proc1.south] {\scriptsize{процесс root}};
        \node[right= 0cm of proc1, minimum width= 0.2cm, minimum height= 0.5cm, align=center] (etc1) {...};

        \node[draw, fill=orange, rectangle, right= 0cm of etc1, minimum width= 2cm, minimum height= 0.5cm] (procN) {};
        \draw[->] ([yshift= 0.3cm] procN.north west) -- (procN.north west);
        \node[above left= 0cm and -0.1cm of procN.north] {\scriptsize{\texttt{buffer}}};
        \draw[<->] ([yshift= -0.2cm] procN.south west) -- ([yshift= -0.2cm] procN.south east);
        \node[fill=white, below= 0cm of procN.south] {\scriptsize{\texttt{count}}};
        \node[below= 0.3cm of procN.south] {\scriptsize{процесс n-1}};

        \path[] (memroot) -- (proc0) node[draw, rotate=-135, pos=0.5, single arrow, minimum height=1.5cm] {};
        \path[] (memroot) -- (proc1) node[draw, rotate=-90, pos=0.5, single arrow, minimum height=1.5cm] {};
        \path[] (memroot) -- (procN) node[draw, rotate=-45, pos=0.5, single arrow, minimum height=1.5cm] {};
    \end{tikzpicture}
\end{figure}

\end{frame}

\section{Функции распределения данных}

\begin{frame}[fragile]{MPI_Scatter}

\begin{lstlisting}
int MPI_Scatter(const void *sendbuf, int sendcount, MPI_Datatype sendtype, void *recvbuf, int recvcount, MPI_Datatype recvtype, int root, MPI_Comm comm);
\end{lstlisting}

\end{frame}

\begin{frame}{MPI_Scatter}

\begin{figure}[htp]
    \centering
    \begin{tikzpicture}
        \node[] (root) {Процесс root};
        \node[below = 2cm of root] (all) {Все процессы};

        \node[draw, fill=orange, rectangle, right= 1cm of root, minimum width= 2cm, minimum height= 0.5cm] (memroot0) {};
        \draw[->] ([yshift= 0.3cm] memroot0.north west) -- (memroot0.north west);
        \node[above left= 0cm and -0.2cm of memroot0.north] {\scriptsize{\texttt{sendbuf}}};
        \draw[<->] ([yshift= -0.2cm] memroot0.south west) -- ([yshift= -0.2cm] memroot0.south east);
        \node[fill=white, below= 0cm of memroot0.south] {\scriptsize{\texttt{sendcount}}};

        \node[draw, fill=orange, rectangle, right= 0cm of memroot0, minimum width= 2cm, minimum height= 0.5cm] (memroot1) {};
        \draw[<->] ([yshift= -0.2cm] memroot1.south west) -- ([yshift= -0.2cm] memroot1.south east);
        \node[fill=white, below= 0cm of memroot1.south] {\scriptsize{\texttt{sendcount}}};
        \node[draw, rectangle, right= 0cm of memroot1, minimum width= 0.2cm, minimum height= 0.5cm, align=center] (etcroot) {...};

        \node[draw, fill=orange, rectangle, right= 0cm of etcroot, minimum width= 2cm, minimum height= 0.5cm] (memrootN) {};
        \draw[<->] ([yshift= -0.2cm] memrootN.south west) -- ([yshift= -0.2cm] memrootN.south east);
        \node[fill=white, below= 0cm of memrootN.south] {\scriptsize{\texttt{sendcount}}};

        \node[draw, fill=orange, rectangle, right= 0.75cm of all, minimum width= 2cm, minimum height= 0.5cm] (memall0) {};
        \draw[->] ([yshift= 0.3cm] memall0.north west) -- (memall0.north west);
        \node[above left= 0cm and -0.2cm of memall0.north] {\scriptsize{\texttt{recvbuf}}};
        \draw[<->] ([yshift= -0.2cm] memall0.south west) -- ([yshift= -0.2cm] memall0.south east);
        \node[fill=white, below= 0cm of memall0.south] {\scriptsize{\texttt{recvcount}}};

        \node[draw, fill=orange, rectangle, right= 0.5cm of memall0, minimum width= 2cm, minimum height= 0.5cm] (memall1) {};
        \draw[->] ([yshift= 0.3cm] memall1.north west) -- (memall1.north west);
        \node[above left= 0cm and -0.2cm of memall1.north] {\scriptsize{\texttt{recvbuf}}};
        \draw[<->] ([yshift= -0.2cm] memall1.south west) -- ([yshift= -0.2cm] memall1.south east);
        \node[fill=white, below= 0cm of memall1.south] {\scriptsize{\texttt{recvcount}}};
        \node[right= 0cm of memall1, minimum width= 0.2cm, minimum height= 0.5cm, align=center] (etcall) {...};

        \node[draw, fill=orange, rectangle, right= 0cm of etcall, minimum width= 2cm, minimum height= 0.5cm] (memallN) {};
        \draw[->] ([yshift= 0.3cm] memallN.north west) -- (memallN.north west);
        \node[above left= 0cm and -0.2cm of memallN.north] {\scriptsize{\texttt{recvbuf}}};
        \draw[<->] ([yshift= -0.2cm] memallN.south west) -- ([yshift= -0.2cm] memallN.south east);
        \node[fill=white, below= 0cm of memallN.south] {\scriptsize{\texttt{recvcount}}};

        \path[] (memroot0) -- (memall0) node[draw, rotate=-90, pos=0.45, single arrow, minimum height=1.2cm] {};
        \path[] (memroot1) -- (memall1) node[draw, rotate=-90, pos=0.45, single arrow, minimum height=1.2cm] {};
        \path[] (memrootN) -- (memallN) node[draw, rotate=-90, pos=0.45, single arrow, minimum height=1.2cm] {};
    \end{tikzpicture}
\end{figure}

\end{frame}

\begin{frame}[fragile]{Пример}

\begin{lstlisting}
MPI_Comm comm;
int  rbuf[100], gsize;
int  root, *array;
// ...
MPI_Comm_size(comm, &gsize);
array = (int *) malloc(gsize * 100 * sizeof(int));
// ...
MPI_Scatter(array, 100, MPI_INT, rbuf, 100, MPI_INT, root, comm);
\end{lstlisting}

\end{frame}

\begin{frame}[fragile]{MPI_Scatterv}

\begin{lstlisting}
int MPI_Scatterv(const void *sendbuf, const int *sendcounts, const int *displs, MPI_Datatype sendtype, void *recvbuf, int recvcount, MPI_Datatype recvtype, int root, MPI_Comm comm);
\end{lstlisting}

\end{frame}

\begin{frame}{MPI_Scatterv}

\begin{figure}[htp]
    \centering
    \begin{tikzpicture}
        \node[] (root) {Процесс root};
        \node[below = 3cm of root] (all) {Все процессы};

        \node[draw, fill=orange, rectangle, right= 0.5cm of root, minimum width= 2cm, minimum height= 0.5cm] (memroot0) {};
        \draw[->] ([yshift= 0.3cm] memroot0.north west) -- (memroot0.north west);
        \node[above left= 0cm and -0.2cm of memroot0.north] {\scriptsize{\texttt{sendbuf}}};
        \node[draw, rectangle, right= 0cm of memroot0, minimum width= 0.2cm, minimum height= 0.5cm, align=center] (etcroot0) {...};

        \node[draw, fill=brown, rectangle, right= 0cm of etcroot0, minimum width= 2.6cm, minimum height= 0.5cm] (memroot1) {};
        \draw[<->] ([yshift= -0.2cm] memroot1.south west) -- ([yshift= -0.2cm] memroot1.south east);
        \node[fill=white, below= 0cm of memroot1.south] {\scriptsize{\texttt{sendcounts[i]}}};
        \node[draw, rectangle, right= 0cm of memroot1, minimum width= 0.2cm, minimum height= 0.5cm, align=center] (etcroot1) {...};

        \node[draw, fill=yellow, rectangle, right= 0cm of etcroot1, minimum width= 2cm, minimum height= 0.5cm] (memrootN) {};

        \draw[-] ([yshift= -0.5cm] memroot0.south west) -- (memroot0.south west);
        \draw[-] ([yshift= -0.5cm] memroot1.south west) -- (memroot1.south west);
        \draw[<->] ([yshift= -0.4cm] memroot0.south west) -- ([yshift= -0.4cm] memroot1.south west);
        \node[fill=white, below right= 0.15cm and -0.5cm of memroot0.south] {\scriptsize{\texttt{displs[i]}}};

        \node[draw, fill=orange, rectangle, right= 0.5cm of all, minimum width= 2cm, minimum height= 0.5cm] (memall0) {};
        \draw[->] ([yshift= 0.3cm] memall0.north west) -- (memall0.north west);
        \node[above left= 0cm and -0.2cm of memall0.north] {\scriptsize{\texttt{recvbuf}}};
        \draw[<->] ([yshift= -0.2cm] memall0.south west) -- ([yshift= -0.2cm] memall0.south east);
        \node[fill=white, below= 0cm of memall0.south] {\scriptsize{\texttt{recvcount}}};
        \node[right= 0cm of memall0, minimum width= 0.2cm, minimum height= 0.5cm, align=center] (etcall0) {...};

        \node[draw, fill=brown, rectangle, right= 0cm of etcall0, minimum width= 2.6cm, minimum height= 0.5cm] (memall1) {};
        \draw[->] ([yshift= 0.3cm] memall1.north west) -- (memall1.north west);
        \node[above left= 0cm and 0.05cm of memall1.north] {\scriptsize{\texttt{recvbuf}}};
        \draw[<->] ([yshift= -0.2cm] memall1.south west) -- ([yshift= -0.2cm] memall1.south east);
        \node[fill=white, below= 0cm of memall1.south] {\scriptsize{\texttt{recvcount}}};
        \node[right= 0cm of memall1, minimum width= 0.2cm, minimum height= 0.5cm, align=center] (etcall1) {...};

        \node[draw, fill=yellow, rectangle, right= 0cm of etcall1, minimum width= 2cm, minimum height= 0.5cm] (memallN) {};
        \draw[->] ([yshift= 0.3cm] memallN.north west) -- (memallN.north west);
        \node[above left= 0cm and -0.2cm of memallN.north] {\scriptsize{\texttt{recvbuf}}};
        \draw[<->] ([yshift= -0.2cm] memallN.south west) -- ([yshift= -0.2cm] memallN.south east);
        \node[fill=white, below= 0cm of memallN.south] {\scriptsize{\texttt{recvcount}}};

        \path[] (memroot0) -- (memall0) node[draw, rotate=-90, pos=0.5, single arrow, minimum height=1.5cm] {};
        \path[] (memroot1) -- (memall1) node[draw, rotate=-90, pos=0.5, single arrow, minimum height=1.5cm] {};
        \path[] (memrootN) -- (memallN) node[draw, rotate=-90, pos=0.5, single arrow, minimum height=1.5cm] {};
    \end{tikzpicture}
\end{figure}

\end{frame}

\section{Функции сбора данных}

\begin{frame}[fragile]{MPI_Gather}

\begin{lstlisting}
int MPI_Gather(const void *sendbuf, int sendcount, MPI_Datatype sendtype, void *recvbuf, int recvcount, MPI_Datatype recvtype, int root, MPI_Comm comm);
\end{lstlisting}

\end{frame}

\begin{frame}{MPI_Gather}

\begin{figure}[htp]
    \centering
    \begin{tikzpicture}
        \node[] (root) {Процесс root};
        \node[below = 2cm of root] (all) {Все процессы};

        \node[draw, fill=orange, rectangle, right= 1cm of root, minimum width= 2cm, minimum height= 0.5cm] (memroot0) {};
        \draw[->] ([yshift= 0.3cm] memroot0.north west) -- (memroot0.north west);
        \node[above left= 0cm and -0.2cm of memroot0.north] {\scriptsize{\texttt{recvbuf}}};
        \draw[<->] ([yshift= -0.2cm] memroot0.south west) -- ([yshift= -0.2cm] memroot0.south east);
        \node[fill=white, below= 0cm of memroot0.south] {\scriptsize{\texttt{recvcount}}};

        \node[draw, fill=orange, rectangle, right= 0cm of memroot0, minimum width= 2cm, minimum height= 0.5cm] (memroot1) {};
        \draw[<->] ([yshift= -0.2cm] memroot1.south west) -- ([yshift= -0.2cm] memroot1.south east);
        \node[fill=white, below= 0cm of memroot1.south] {\scriptsize{\texttt{recvcount}}};
        \node[draw, rectangle, right= 0cm of memroot1, minimum width= 0.2cm, minimum height= 0.5cm, align=center] (etcroot) {...};

        \node[draw, fill=orange, rectangle, right= 0cm of etcroot, minimum width= 2cm, minimum height= 0.5cm] (memrootN) {};
        \draw[<->] ([yshift= -0.2cm] memrootN.south west) -- ([yshift= -0.2cm] memrootN.south east);
        \node[fill=white, below= 0cm of memrootN.south] {\scriptsize{\texttt{recvcount}}};

        \node[draw, fill=orange, rectangle, right= 0.75cm of all, minimum width= 2cm, minimum height= 0.5cm] (memall0) {};
        \draw[->] ([yshift= 0.3cm] memall0.north west) -- (memall0.north west);
        \node[above left= 0cm and -0.2cm of memall0.north] {\scriptsize{\texttt{sendbuf}}};
        \draw[<->] ([yshift= -0.2cm] memall0.south west) -- ([yshift= -0.2cm] memall0.south east);
        \node[fill=white, below= 0cm of memall0.south] {\scriptsize{\texttt{sendcount}}};

        \node[draw, fill=orange, rectangle, right= 0.5cm of memall0, minimum width= 2cm, minimum height= 0.5cm] (memall1) {};
        \draw[->] ([yshift= 0.3cm] memall1.north west) -- (memall1.north west);
        \node[above left= 0cm and -0.2cm of memall1.north] {\scriptsize{\texttt{sendbuf}}};
        \draw[<->] ([yshift= -0.2cm] memall1.south west) -- ([yshift= -0.2cm] memall1.south east);
        \node[fill=white, below= 0cm of memall1.south] {\scriptsize{\texttt{sendcount}}};
        \node[right= 0cm of memall1, minimum width= 0.2cm, minimum height= 0.5cm, align=center] (etcall) {...};

        \node[draw, fill=orange, rectangle, right= 0cm of etcall, minimum width= 2cm, minimum height= 0.5cm] (memallN) {};
        \draw[->] ([yshift= 0.3cm] memallN.north west) -- (memallN.north west);
        \node[above left= 0cm and -0.2cm of memallN.north] {\scriptsize{\texttt{sendbuf}}};
        \draw[<->] ([yshift= -0.2cm] memallN.south west) -- ([yshift= -0.2cm] memallN.south east);
        \node[fill=white, below= 0cm of memallN.south] {\scriptsize{\texttt{sendcount}}};

        \path[] (memroot0) -- (memall0) node[draw, rotate=90, pos=0.5, single arrow, minimum height=1.2cm] {};
        \path[] (memroot1) -- (memall1) node[draw, rotate=90, pos=0.5, single arrow, minimum height=1.2cm] {};
        \path[] (memrootN) -- (memallN) node[draw, rotate=90, pos=0.5, single arrow, minimum height=1.2cm] {};
    \end{tikzpicture}
\end{figure}

\end{frame}

\begin{frame}[fragile]{MPI_Gatherv}

\begin{lstlisting}
int MPI_Gatherv(void* sendbuf, int sendcount, MPI_Datatype sendtype, void* rbuf, int *recvcounts, int *displs, MPI_Datatype recvtype, int root, MPI_Comm comm);
\end{lstlisting}

\end{frame}

\begin{frame}{MPI_Gatherv}

\begin{figure}[htp]
    \centering
    \begin{tikzpicture}
        \node[] (root) {Процесс root};
        \node[below = 3cm of root] (all) {Все процессы};

        \node[draw, fill=orange, rectangle, right= 0.5cm of root, minimum width= 2cm, minimum height= 0.5cm] (memroot0) {};
        \draw[->] ([yshift= 0.3cm] memroot0.north west) -- (memroot0.north west);
        \node[above left= 0cm and -0.2cm of memroot0.north] {\scriptsize{\texttt{recvbuf}}};
        \node[draw, rectangle, right= 0cm of memroot0, minimum width= 0.2cm, minimum height= 0.5cm, align=center] (etcroot0) {...};

        \node[draw, fill=brown, rectangle, right= 0cm of etcroot0, minimum width= 2.6cm, minimum height= 0.5cm] (memroot1) {};
        \draw[<->] ([yshift= -0.2cm] memroot1.south west) -- ([yshift= -0.2cm] memroot1.south east);
        \node[fill=white, below= 0cm of memroot1.south] {\scriptsize{\texttt{recvcounts[i]}}};
        \node[draw, rectangle, right= 0cm of memroot1, minimum width= 0.2cm, minimum height= 0.5cm, align=center] (etcroot1) {...};

        \node[draw, fill=yellow, rectangle, right= 0cm of etcroot1, minimum width= 2cm, minimum height= 0.5cm] (memrootN) {};

        \draw[-] ([yshift= -0.5cm] memroot0.south west) -- (memroot0.south west);
        \draw[-] ([yshift= -0.5cm] memroot1.south west) -- (memroot1.south west);
        \draw[<->] ([yshift= -0.4cm] memroot0.south west) -- ([yshift= -0.4cm] memroot1.south west);
        \node[fill=white, below right= 0.15cm and -0.5cm of memroot0.south] {\scriptsize{\texttt{displs[i]}}};

        \node[draw, fill=orange, rectangle, right= 0.5cm of all, minimum width= 2cm, minimum height= 0.5cm] (memall0) {};
        \draw[->] ([yshift= 0.3cm] memall0.north west) -- (memall0.north west);
        \node[above left= 0cm and -0.2cm of memall0.north] {\scriptsize{\texttt{sendbuf}}};
        \draw[<->] ([yshift= -0.2cm] memall0.south west) -- ([yshift= -0.2cm] memall0.south east);
        \node[fill=white, below= 0cm of memall0.south] {\scriptsize{\texttt{sendcount}}};
        \node[right= 0cm of memall0, minimum width= 0.2cm, minimum height= 0.5cm, align=center] (etcall0) {...};

        \node[draw, fill=brown, rectangle, right= 0cm of etcall0, minimum width= 2.6cm, minimum height= 0.5cm] (memall1) {};
        \draw[->] ([yshift= 0.3cm] memall1.north west) -- (memall1.north west);
        \node[above left= 0cm and 0.05cm of memall1.north] {\scriptsize{\texttt{sendbuf}}};
        \draw[<->] ([yshift= -0.2cm] memall1.south west) -- ([yshift= -0.2cm] memall1.south east);
        \node[fill=white, below= 0cm of memall1.south] {\scriptsize{\texttt{sendcount}}};
        \node[right= 0cm of memall1, minimum width= 0.2cm, minimum height= 0.5cm, align=center] (etcall1) {...};

        \node[draw, fill=yellow, rectangle, right= 0cm of etcall1, minimum width= 2cm, minimum height= 0.5cm] (memallN) {};
        \draw[->] ([yshift= 0.3cm] memallN.north west) -- (memallN.north west);
        \node[above left= 0cm and -0.2cm of memallN.north] {\scriptsize{\texttt{sendbuf}}};
        \draw[<->] ([yshift= -0.2cm] memallN.south west) -- ([yshift= -0.2cm] memallN.south east);
        \node[fill=white, below= 0cm of memallN.south] {\scriptsize{\texttt{sendcount}}};

        \path[] (memroot0) -- (memall0) node[draw, rotate=90, pos=0.5, single arrow, minimum height=1.5cm] {};
        \path[] (memroot1) -- (memall1) node[draw, rotate=90, pos=0.5, single arrow, minimum height=1.5cm] {};
        \path[] (memrootN) -- (memallN) node[draw, rotate=90, pos=0.5, single arrow, minimum height=1.5cm] {};
    \end{tikzpicture}
\end{figure}

\end{frame}

\section*{Конец}
% The final "thank you" frame 

\begin{frame}{Задания}

Реализовать параллельную сортировку массива.

Имя файла задается через аргументы командной строки.

Построить графики зависимости времени работы и ускорения от количества процессов.

\end{frame}

\begin{frame}{На следующей лекции}
\end{frame}

\begin{frame}

{\huge{Спасибо за внимание!}\par}

\vfill

\tiny{\textit{Замечание}: все торговые марки и логотипы, использованные в данном материале, являются собственностью их владельцев. Представленная здесь точка зрения отражает личное мнение автора, не выступающего от лица какой-либо организации.}

\end{frame}

\end{document}
