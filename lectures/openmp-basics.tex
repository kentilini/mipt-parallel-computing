% Compile with XeLaTeX, TeXLive 2013 or more recent
\documentclass{beamer}

% Base packages
\usepackage{fontspec}

\usepackage{xunicode}
\usepackage{xltxtra}

\usepackage{amsfonts}
\usepackage{amsmath}
\usepackage{longtable}
\usepackage{csquotes}
\usepackage{standalone}

\usepackage{graphicx}
\graphicspath{{./images/}}

\usepackage{tikz}
\usetikzlibrary{arrows,decorations.pathmorphing,backgrounds,positioning,fit,petri}

\usepackage{listings}
\lstset{language=C, basicstyle=\ttfamily, breaklines=true, keepspaces=true, keywordstyle=\color{blue}}

% Setup Russian hyphenation
\usepackage{polyglossia}
\setdefaultlanguage[spelling=modern]{russian} % for polyglossia
\setotherlanguage{english} % for polyglossia

% Setup fonts
\newfontfamily\russianfont{CMU Serif}
\setromanfont{CMU Serif}
\setsansfont{CMU Sans Serif}
\setmonofont{CMU Typewriter Text}

% Be able to insert hyperlinks
\usepackage{hyperref}
\hypersetup{colorlinks=true, linkcolor=black, filecolor=black, citecolor=black, urlcolor=blue , pdfauthor=Evgeny Yulyugin <yulyugin@gmail.com>, pdftitle=Параллельное программирование}
% \usepackage{url}

% Misc optional packages
\usepackage{underscore}
\usepackage{amsthm}

% A new command to mark not done places
\newcommand{\todo}[1][]{{\color{red}TODO\ #1}}

\newcommand{\abbr}{\textit{англ.}\ }

\subtitle{Курс «Параллельное программирование»}
\subject{Lecture}
\author[Евгений Юлюгин]{Евгений Юлюгин \\ \small{\href{mailto:yulyugin@gmail.com}{yulyugin@gmail.com}}}
\date{\today}
\pgfdeclareimage[height=0.5cm]{mipt-logo}{../common/mipt.png}
\logo{\pgfuseimage{mipt-logo}}

\typeout{Copyright 2014 Evgeny Yulyugin}

\usetheme{Berlin}
\setbeamertemplate{navigation symbols}{}%remove navigation symbols


\title{Основы OpenMP}

\begin{document}

\section*{Обзор}

\begin{frame}
\titlepage
\end{frame}

\begin{frame}{На этой лекции}
\tableofcontents
\end{frame}

% Use [fragile] option to insert listings
% \begin{frame}[fragile]

\section{История OpenMP}

\begin{frame}

OpenMP (Open Multi-Processing) --- открытый стандарт для распараллеливания программ на языках C, C++, Fortran.

Описывает совокупность директив компилятора, процедур и переменных окружения.

1997 год --- первая версия для Fortran.

1998 год --- первая версия для C/C++.

\end{frame}

\begin{frame}[fragile]{Директивы компилятора}

Директива компилятора --- указание особенностей обработки при компиляции.

Директивы препроцессора:

\begin{lstlisting}
#include "file"
#define
\end{lstlisting}

Директива \texttt{\#pragma} --- поддержка функций уникальных для хоста или компилятора (ОС) с сохранением совместимости языков C/C++.

\end{frame}

\begin{frame}{Примеры использования \#pragma}

\texttt{\#pragma once} --- аналог include guard.

\texttt{\#pragma message} --- определяет сообщение выдоваемое при компиляции.

\texttt{\#pragma omp}

\end{frame}

\section{Основные команды OpenMP}

\begin{frame}[fragile]{Директивы OpenMP}

\begin{lstlisting}
#pragma omp <directive> options
\end{lstlisting}

\begin{itemize}
    \item parallel,
    \item for,
    \item single,
    \item section,
    \item critical,
    \item atomic,
    \item barrier.
\end{itemize}

\end{frame}

\begin{frame}[fragile]{Директивы parallel и for}

Директива \texttt{parallel} сообщает компилятору, что блок кода должен быть выполнен в нескольких потоках.

Директива \texttt{for} распределяет итерации цикла между тредами.

\end{frame}

\begin{frame}{Некоторые опции}

\begin{itemize}
    \item if --- исполнить параллельно, если истина. Последовательно в противном случае.
    \item private(list) --- локальные для потоков переменные. При входе и выходе значения не определены.
    \item shared(list) --- общение для всех потоков переменные.
\end{itemize}

\end{frame}

\begin{frame}[fragile]{Пример}
\begin{lstlisting}
int i, N = 8;
#pragma omp parallel private(i)
{
    for (i = 0; i < N; i++)
        printf("%d: %d\n",
               omp_get_thread_num(), i);
}
\end{lstlisting}

\pause

Что неправильно?
\end{frame}

\begin{frame}[fragile]{Пример}
\begin{lstlisting}
int i, N = 8;
#pragma omp parallel private(i)
{
    int id = omp_get_thread_num();
    for (i = 0; i < N; i++)
        printf("%d: %d\n", id, i);
}
\end{lstlisting}

\pause

Что будет напечатано?
\end{frame}

\begin{frame}[fragile]{Пример}
\begin{lstlisting}
int i, N = 8;
#pragma omp parallel for
{
    int id = omp_get_thread_num();
    for (i = 0; i < N; i++)
        printf("%d: %d\n", id, i);
}
\end{lstlisting}
\end{frame}

\begin{frame}[fragile]{Пример}
\begin{lstlisting}
int i, N = atoi(argv[1]);
#pragma omp parallel if(N >= 8) private(i)
{
    int id = omp_get_thread_num();
    for (i = 0; i < N; i++)
        printf("%d: %d\n", id, i);
}
\end{lstlisting}
\end{frame}

\begin{frame}[fragile]{Процедуры OpenMP}

\begin{itemize}
    \item Возвращает номер потока
    \begin{lstlisting}
int omp_get_thread_num()
    \end{lstlisting}

    \item Определяет количество потоков используемое по-умолчанию в параллельной секции.
    \begin{lstlisting}
void omp_set_num_threads(int num_threads);
    \end{lstlisting}
\end{itemize}

\end{frame}

\begin{frame}[fragile]{Пример}

\begin{lstlisting}
#pragma omp parallel
{
    myid = omp_get_thread_num();
    if (myid == 0) {
        <...>
    } else {
        <...>
    }
}
\end{lstlisting}

\end{frame}

\begin{frame}{Переменные окружения}
\texttt{OMP_NUM_THREADS} --- определяет количество потоков используемое по-умолчанию в параллельной секции.
\end{frame}

\begin{frame}[fragile]

\begin{lstlisting}
#pragma omp for
\end{lstlisting}

Будет выполнена с уже созданными потоками, что позволит сохранить время.

\end{frame}

\section{Редукция}

\begin{frame}{Опция reduction}
Формат опции: reduction(оператор: список)

\bigskip

Поддерживаеые операторы: +, *, -, \&, |, \^~, \&\&, ||.

\bigskip

Переменные должны иметь скалярный тип: \texttt{float}, \texttt{int},
\texttt{std::vector}, и т.д.
\end{frame}

\begin{frame}{Принцип работы}
\begin{enumerate}
    \item Для каждой переменной создаются локальные копии в каждом потоке;
    \item Переменные инициализируются: 0 для аддитивных операций и 1 для
    мултипликативных;
    \item После завершения над локальными переменными выполняется заданный
    оператор.
\end{enumerate}
\end{frame}

\begin{frame}{Операторы reduction}
\begin{table}[htpb]
    \centering
    \begin{tabular}{|c|c|}
    \hline
    Оператор    &   Исходное значение   \\
    \hline
    +           &   0                   \\
    \hline
    *           &   1                   \\
    \hline
    -           &   0                   \\
    \hline
    \&          &   \~~0                \\
    \hline
    |           &   0                   \\
    \hline
    \^~         &   0                   \\
    \hline
    \&\&        &   1                   \\
    \hline
    ||          &   0                   \\
    \hline
    \end{tabular}
\end{table}
\end{frame}

\begin{frame}[fragile]{Пример}

Вычисление суммы элементов массива:

\begin{lstlisting}
#pragma omp parallel for reduction(+:sum)
    for (i = 0; i < N; i++)
        sum += a[i];
\end{lstlisting}

\end{frame}

\begin{frame}[fragile]{Динамическое распределение нагрузки}

Опция \texttt{schedule}

\begin{lstlisting}[basicstyle=\scriptsize]
#pragma omp parallel for shared(a), schedule(dynamic)
    for (i = 0; i < N; i++)
        a[i] = a[i] * a[i];
\end{lstlisting}

\end{frame}

\begin{frame}[fragile]{Директива single}

\begin{lstlisting}
#pragma omp single
\end{lstlisting}

Обозначение блоков внутри параллельной секции, которые должны быть выполнены
один раз.

\bigskip

\pause

Опция \texttt{copyprivate(list)} определяет \texttt{private} переменные которые
будут иметь обновленное значение после выполнения блок \texttt{single}.

\end{frame}

\begin{frame}[fragile]{Пример}

\begin{lstlisting}
#pragma omp parallel
{
    #pragma omp for
    {
        for (...)
    }
    #pragma omp single
    {
        write_result(...);
    }
}
\end{lstlisting}

\end{frame}

\begin{frame}[fragile]{Секции}

\begin{lstlisting}
#pragma omp parallel
{
    #pragma omp sections
        init();
    #pragma omp section
        task_1();
    #pragma omp section
        task_2();
}
\end{lstlisting}

\end{frame}

\section{Синхронизация в OpenMP}

\begin{frame}[fragile]

\begin{lstlisting}
#pragma omp critical (label)
{
    <...>
}
\end{lstlisting}

\end{frame}

\begin{frame}[fragile]

\begin{lstlisting}
#pragma omp atomic
{
    sum += value;
}
\end{lstlisting}

\end{frame}

\begin{frame}[fragile]

\begin{lstlisting}
omp_lock_t lock;
omp_init_lock(&lock);

#pragma omp parallel
{
    omp_set_lock(&lock);
    // ...
    omp_unset_lock(&lock);
    omp_test_lock(&lock);
    omp_destroy_lock(&lock);
}
\end{lstlisting}

\end{frame}

\begin{frame}[fragile]{Директива ordered}

\begin{lstlisting}
#pragma parallel private(i)
{
    int id = omp_get_thread_num();

    #pragma omp for ordered
    for (i = 0; i < N; i++) {
        printf("Thread %d, iteration %d\n", id, i);

        #pragma omp ordered
            printf("Thread %d, iteration %d\n", id, i);
    }
}
\end{lstlisting}

\end{frame}

\begin{frame}[fragile]{Измерение времени}

\begin{lstlisting}
omp_set_num_threads();
omp_get_wtime();
\end{lstlisting}

\end{frame}

\section{Компиляция}

\begin{frame}[fragile]{Сборка и запуск}

Поддерживается компилятором начиная с GCC 4.2.

Заголовочный файл \texttt{omp.h}:

\begin{lstlisting}
#include <omp.h>
\end{lstlisting}

\begin{lstlisting}
> gcc -fopenmp file.c
\end{lstlisting}

\end{frame}

\section*{Конец}

\begin{frame}{Задания}

Посчитать интеграл sin(1/x).

\end{frame}

\begin{frame}{На следующей лекции}
\end{frame}

\begin{frame}

{\huge{Спасибо за внимание!}\par}

\vfill

\tiny{\textit{Замечание}: все торговые марки и логотипы, использованные в данном материале, являются собственностью их владельцев. Представленная здесь точка зрения отражает личное мнение автора, не выступающего от лица какой-либо организации.}

\end{frame}

\end{document}
