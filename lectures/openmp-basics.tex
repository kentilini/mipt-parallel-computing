% Compile with XeLaTeX, TeXLive 2013 or more recent
\documentclass{beamer}

% Base packages
\usepackage{fontspec}

\usepackage{xunicode}
\usepackage{xltxtra}

\usepackage{amsfonts}
\usepackage{amsmath}
\usepackage{longtable}
\usepackage{csquotes}
\usepackage{standalone}

\usepackage{graphicx}
\graphicspath{{./images/}}

\usepackage{tikz}
\usetikzlibrary{arrows,decorations.pathmorphing,backgrounds,positioning,fit,petri}

\usepackage{listings}
\lstset{language=C, basicstyle=\ttfamily, breaklines=true, keepspaces=true, keywordstyle=\color{blue}}

% Setup Russian hyphenation
\usepackage{polyglossia}
\setdefaultlanguage[spelling=modern]{russian} % for polyglossia
\setotherlanguage{english} % for polyglossia

% Setup fonts
\newfontfamily\russianfont{CMU Serif}
\setromanfont{CMU Serif}
\setsansfont{CMU Sans Serif}
\setmonofont{CMU Typewriter Text}

% Be able to insert hyperlinks
\usepackage{hyperref}
\hypersetup{colorlinks=true, linkcolor=black, filecolor=black, citecolor=black, urlcolor=blue , pdfauthor=Evgeny Yulyugin <yulyugin@gmail.com>, pdftitle=Параллельное программирование}
% \usepackage{url}

% Misc optional packages
\usepackage{underscore}
\usepackage{amsthm}

% A new command to mark not done places
\newcommand{\todo}[1][]{{\color{red}TODO\ #1}}

\newcommand{\abbr}{\textit{англ.}\ }

\subtitle{Курс «Параллельное программирование»}
\subject{Lecture}
\author[Евгений Юлюгин]{Евгений Юлюгин \\ \small{\href{mailto:yulyugin@gmail.com}{yulyugin@gmail.com}}}
\date{\today}
\pgfdeclareimage[height=0.5cm]{mipt-logo}{../common/mipt.png}
\logo{\pgfuseimage{mipt-logo}}

\typeout{Copyright 2014 Evgeny Yulyugin}

\usetheme{Berlin}
\setbeamertemplate{navigation symbols}{}%remove navigation symbols


\title{Основы OpenMP}

\begin{document}

\section*{Обзор}

\begin{frame}
\titlepage
\end{frame} 

\begin{frame}{На этой лекции}
\tableofcontents
\end{frame}

% Use [fragile] option to insert listings
% \begin{frame}[fragile]

\section{История OpenMP}

\begin{frame}

OpenMP (Open Multi-Processing) --- открытый стандарт для распараллеливания программ на языках C, C++, Fortran.

Описывает совокупность директив компилятора, процедур и переменных окружения.

1997 год --- первая версия для Fortran.

1998 год --- первая версия для C/C++.

\end{frame}

\begin{frame}[fragile]{Директивы компилятора}

Директива компилятора --- указание особенностей обработки при компиляции.

Директивы препроцессора:

\begin{lstlisting}
#include "file"
#define
\end{lstlisting}

Директива \texttt{\#pragma} --- поддержка функций уникальных для хоста или компилятора (ОС) с сохранением совместимости языков C/C++.

\end{frame}

\begin{frame}{Примеры использования \#pragma}

\texttt{\#pragma once} --- аналог include guard.

\texttt{\#pragma message} --- определяет сообщение выдоваемое при компиляции.

\texttt{\#pragma omp}

\end{frame}

\section{Основные команды OpenMP}

\begin{frame}[fragile]{Директивы OpenMP}

\begin{lstlisting}
#pragma omp <directive> options
\end{lstlisting}

\begin{itemize}
    \item parallel,
    \item for,
    \item single,
    \item section,
    \item critical,
    \item atomic,
    \item barrier.
\end{itemize}

\end{frame}

\begin{frame}[fragile]{Директивы parallel и for}

Директива \texttt{parallel} сообщает компилятору, что блок кода должен быть выполнен в нескольких потоках.

Директива \texttt{for} распределяет итерации цикла между тредами.

\end{frame}

\begin{frame}{Некоторые опции}

\begin{itemize}
    \item if --- исполнить параллельно, если истина. Последовательно в противном случае.
    \item private(list) --- локальные для потоков переменные. При входе и выходе значения не определены.
    \item shared(list) --- общение для всех потоков переменные.
\end{itemize}

\end{frame}

\begin{frame}[fragile]{Пример}

\begin{lstlisting}
int a = {...};
int j = 0;

#pragma omp parallel
{
  for (j = 0; j < N; ++j)
    a[j] = a[j] * a[j];
}
\end{lstlisting}

\end{frame}

\begin{frame}[fragile]{Пример}

\begin{lstlisting}
int a = {...};
int j = 0;

#pragma omp parallel for
{
  for (j = 0; j < N; ++j)
    a[j] = a[j] * a[j];
}
\end{lstlisting}

\end{frame}

\begin{frame}[fragile]{Процедуры OpenMP}

\begin{itemize}
    \item Возвращает номер потока
    \begin{lstlisting}
int omp_get_thread_num()
    \end{lstlisting}

    \item Определяет количество потоков используемое по-умолчанию в параллельной секции.
    \begin{lstlisting}
void omp_set_num_threads(int num_threads);
    \end{lstlisting}
\end{itemize}

\end{frame}

\begin{frame}[fragile]{Пример}

\begin{lstlisting}
#pragma omp parallel
{
  myid = omp_get_thread_num();
  if (myid == 0) {
    // ...
  } else {
    // ...
  }
}
\end{lstlisting}

\end{frame}

\begin{frame}{Переменные окружения}
\texttt{OMP_NUM_THREADS} --- определяет количество потоков используемое по-умолчанию в параллельной секции.
\end{frame}

\begin{frame}[fragile]

\begin{lstlisting}
#pragma omp for
\end{lstlisting}

Будет выполнена с уже созданными потоками, что позволит сохранить время.

\end{frame}

\begin{frame}[fragile]{Редукция}

\begin{lstlisting}[language=C++,basicstyle=\ttfamily,keywordstyle=\color{blue},basicstyle=\scriptsize]
#pragma omp parallel reduction (+:sum)
{
  #pragma omp for
  {
    sum+= f(a[j]);
  }
}
\end{lstlisting}

\end{frame}

\begin{frame}[fragile]{Секции}

\begin{lstlisting}[language=C++,basicstyle=\ttfamily,keywordstyle=\color{blue},basicstyle=\scriptsize]
#pragma omp sections
  init();
#pragma omp section
  task_1();
#pragma omp section
  task_2();
\end{lstlisting}

\end{frame}

\section{Синхронизация в OpenMP}

\begin{frame}[fragile]

\begin{lstlisting}[language=C++,basicstyle=\ttfamily,keywordstyle=\color{blue},basicstyle=\scriptsize]
#pragma omp critical (label)
{
    <...>
}
\end{lstlisting}

\end{frame}

\begin{frame}[fragile]

\begin{lstlisting}[language=C++,basicstyle=\ttfamily,keywordstyle=\color{blue},basicstyle=\scriptsize]
#pragma omp atomic
{
  sum += value;
}
\end{lstlisting}

\end{frame}

\begin{frame}[fragile]

\begin{lstlisting}[language=C++,basicstyle=\ttfamily,keywordstyle=\color{blue},basicstyle=\scriptsize]
omp_lock_t losk;
omp_init_lock(&lock);

#pragma omp parallel
{
  omp_set_lock(&lock);
  // ...
  omp_unset_lock(&lock);
  omp_test_lock(&lock);
  omp_destroy_lock(&lock);
}
\end{lstlisting}

\end{frame}

\begin{frame}[fragile]

\begin{lstlisting}[language=C++,basicstyle=\ttfamily,keywordstyle=\color{blue},basicstyle=\scriptsize]
#pragma omp parallel
{
  while (i < N)
  {
    #pragma omp for
    {
      for (...)
    }
    #pragma omp single
    {
      ++i;
    }
  }
}
\end{lstlisting}

\end{frame}

\begin{frame}[fragile]{Динамический режим}

\begin{lstlisting}[language=C++,basicstyle=\ttfamily,keywordstyle=\color{blue},basicstyle=\scriptsize]
#pragma omp parallel for shedul(dinamic)
{
  for (i = 0; i < N; ++i) {
    // ...
  }
}
\end{lstlisting}

\end{frame}

\begin{frame}[fragile]

\begin{lstlisting}[language=C++,basicstyle=\ttfamily,keywordstyle=\color{blue},basicstyle=\scriptsize]
#pragma omp parallel for ordered
{
  for (i = 0; i < N; ++i) {
    // ...
  }
  # pragma omp ordered
  {
    write_result(...);
  }
}
\end{lstlisting}

\end{frame}

\begin{frame}[fragile]{Измерение времени}

\begin{lstlisting}
omp_set_num_threads();
omp_get_wtime();
\end{lstlisting}

\end{frame}

\section{Компиляция}

\begin{frame}[fragile]{Сборка и запуск}

Поддерживается компилятором начиная с GCC 4.2.

Заголовочный файл \texttt{omp.h}:

\begin{lstlisting}
#include <omp.h>
\end{lstlisting}

\begin{lstlisting}
> gcc -fopenmp file.c
\end{lstlisting}

\end{frame}

\section*{Конец}
% The final "thank you" frame 

\begin{frame}{Задания}

Посчитать интеграл sin(1/x).

\end{frame}

\begin{frame}{На следующей лекции}
\end{frame}

\begin{frame}

{\huge{Спасибо за внимание!}\par}

\vfill

\tiny{\textit{Замечание}: все торговые марки и логотипы, использованные в данном материале, являются собственностью их владельцев. Представленная здесь точка зрения отражает личное мнение автора, не выступающего от лица какой-либо организации.}

\end{frame}

\end{document}
